\documentclass{article}

\usepackage[T1]   {fontenc}
\usepackage[utf8] {inputenc}

\usepackage{acmetoolbox}

\usepackage[english]{babel}
\usepackage{blindtext}

\let\type\texttt
\def\macro#1{\texttt{\char`\\#1}}

\parindent0pt

\begin{document}


\section{\macro{getrawparameters} and \macro{getraweparameters}}

No need to declare family first.

\def\toto{toto}
\getrawparameters[test]{tata=\toto}
\getraweparameters{tata=\toto}
\def\toto{tata}

\begin{itemize}
\item \macro{getrawparameters} :  tata/\testtata
\item \macro{getraweparameters} : toto/\tata
\end{itemize}


\section{\macro{getparameters} and \macro{geteparameters}}

Must declare family first.

\pgfkeys{/parameters/declare family=toto}
\pgfkeys{/eparameters/declare family=etoto,
  /eparameters/etoto/titi/.append to=\etototiti,
  /eparameters/etoto/titi/.prepend code={hello}}

hellohello/%
\def\toto{toto}%
\getparameters[toto]{titi=\toto}%
\defaultsep{, }%
\geteparameters[etoto]{titi=\toto, titi=tata}
\def\toto{tata}

\begin{itemize}
\item \macro{getparameters} :  tata/\tototiti
\item \macro{geteparameters} : toto, tata/\etototiti
\end{itemize}


\section{\macro{processcommalist}}

\def\putparen#1{(#1)}
(one)(two)(three=four)/%
\processcommalist[one, two, three=four]\putparen


\clearpage
\section{\macro{emboxit}}

\fbox{%
  \emboxit [width=0.85\textwidth]{
    \hbox{\bf vmode} \blindtext}
  \flushnextbox}

\fbox{%
  \emboxit[width=0.85\textwidth,  vmode=no]{
    \hbox{\bf hmode} \blindtext}
  \flushnextbox}

\fbox{%
  \emboxit[width=0.85\textwidth,  vmode=no]{%
    \hbox{\bf hmode} \blindtext}
  \flushnextbox}

\blank[line]
See the difference. Also note that if in hmode, you need to take care of
spurious spaces.


\clearpage
\section{\macro{wrapit}}

\fbox{%
  \wrapit[width=200pt, allwidth=0.8\textwidth,
          allheight=0.5\textheight, south west]{%
    \hbox{\bf vmode} \blindtext}
  \flushnextbox}


\newdimen\margin\margin15pt
\fbox{%
  \wrapit[width=\textwidth-2\margin,
          left=\hskip\margin, right=\hskip\margin,
          top=\vskip\margin,  bottom=\vskip\margin, vmode=no]{%
      \hbox{\bf hmode} Very small paragraph, barely a single line.}
  \flushnextbox}


\def\margin{15pt}
\fbox{%
  \wrapit[width=\textwidth-\margin*2, % (syntax is different when using a
                                      %  command instead of a dimension)
          left=\hskip\margin, right=\hskip\margin,
          top=\vskip\margin,  bottom=\vskip\margin]{
    \hbox{\bf vmode} Very small paragraph, barely a single line.}
  \flushnextbox}


\clearpage
\section{\macro{wrapbox}}

\setbox42\vbox{\hsize=200pt\blindtext}

\fbox{%
  \wrapbox[width=0.8\textwidth, height=0.5\textheight, south west]\box42
  \flushnextbox}


\newdimen\margin\margin15pt
\setbox42=\vbox{%
  \hsize=\dimexpr\textwidth-2\margin\relax
  Very small paragraph, barely a single line.}

\fbox{%
  \wrapbox[%
    width=\textwidth,
    left=\hskip\margin, right=\hskip\margin,
    top=\vskip\margin, bottom=\vskip\margin]\box42
  \flushnextbox}

\def\margin{15pt}
\fbox{%
  \wrapbox[%
    width=\textwidth,
    left=\hskip\margin, right=\hskip\margin,
    top=\vskip\margin, bottom=\vskip\margin]
  \vbox{%
    \hsize=\dimexpr\textwidth-\margin*2\relax % (syntax is different when using a
                                              %  command instead of a dimension)
    Very small paragraph, barely a single line.}
  \flushnextbox}


\clearpage
\section{\macro{framed}/\type{framedtext} (environment)}

\makeatletter

No options\par
\framed{toto et titi}
\framed{toto et titi et tutu}

\texttt{width=70pt}\par
\framed[width=70pt]{toto et titi}
\framed[width=70pt]{toto et titi et tutu}

\texttt{minwidth=70pt}\par
\framed[minwidth=70pt]{toto et titi}
\framed[minwidth=70pt]{toto et titi et tutu}

\texttt{maxwidth=70pt}\par
\framed[maxwidth=70pt]{toto et titi}
\framed[maxwidth=70pt]{toto et titi et tutu}

\blank[2*line]
\newcommand\crossout[1][]{%
  \framed[%
    offset=0pt, frame=off, background=on,
    background/custom={
      \begin{pgfonlayer}{foreground}
        \path [draw=red, line cap=round, very thick, #1]
          (0,0) -- (\framedboxwd,-\framedboxht)
          (0,-\framedboxht) -- (\framedboxwd,0);
      \end{pgfonlayer}}]}

\leavevmode
\crossout{this is wrong} and this is right

\blank[line]
\crossout[line width=20pt]{\blindtext}

\blank[2*line]
Available options\par
\begin{itemize}
\item lengths: %
  \texttt{width, minwidth, maxwidth, height, minheight, maxheight};
\item lengths with left/right/top/bottom prefix: %
  \texttt{margin, offset};
\item \texttt{align}:\\
  horizontal \texttt{flushleft/flushright/center}\\
  vertical \texttt{top/bottom/middle} or any combination;
\item \texttt{vmode}: \texttt{yes/no} to start in vmode or not;
\item \texttt{frame, background}: \texttt{on/off} and also
  \texttt{frame/custom} to give your own tikz path (also background);
\item \texttt{options}: passed to tikzpicture environment.
\end{itemize}


\end{document}
